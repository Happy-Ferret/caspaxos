\documentclass[12pt]{article}

\usepackage[toc,page]{appendix}

\usepackage[utf8]{inputenc}

\usepackage{etoolbox}
\AtBeginEnvironment{quote}{\small}

\usepackage{fontspec}
\setmainfont[Ligatures=TeX]{Liberation Serif}
\setsansfont{Liberation Sans}
\setmonofont{Liberation Mono}

\usepackage{mathtools}
\DeclarePairedDelimiter{\floor}{\lfloor}{\rfloor}

\usepackage{polyglossia}
\setdefaultlanguage{english}

\usepackage[colorlinks=true,urlcolor=blue,linkcolor=blue]{hyperref}

\begin{document}

\title{CASPaxos: Replicated State Machines without logs}
\date{30 January 2018}
\author{Denis Rystsov \\
\href{mailto:rystsov.denis@gmail.com}{rystsov.denis@gmail.com}}


\maketitle

\begin{abstract}
CASPaxos is a Single Decree Paxos (Synod) based protocol for building a replicated state machine. Unlike Raft and Multi-Paxos, it doesn't use an idea of the leader election and the replicated log, so it avoids associated complexity (e.g. log compaction) and provides graceful performance degradation when replicas are crashed.

The lightweight nature of CASPaxos opens ways for a new application for RSM, e.g. instead of representing a key/value storage as a single RSM; it's possible to run an instance of CASPaxos per key and achieve a result similar to EPaxos.

This paper describes CASPaxos protocol, proves its safety properties, covers cluster membership change and evaluates a CASPaxos-based key/value storage against established consistent databases.
\end{abstract}


\section{Introduction}

Multi-Paxos\cite{lamport01} and Raft\cite{raft} protocols allow a collection of machines to work as a coherent group - a replicated state machine (RSM). They are proven to preserve liveness when at least $\floor{N/2} + 1$ of $N$ machines are alive and connected, and to preserve safety in the presence of arbitrary crash/recovery of machines and loss of messages.

The problem of keeping RSM work when its nodes are falling apart is isomorphic to the problem of master-master replication of linearizable distributed storage in the presence of arbitrary crashes. So those protocols are widely used in the idustry as a foundation of databases such as Chubby\cite{chubby}, Etcd\footnote{\href{https://github.com/coreos/etcd}{https://github.com/coreos/etcd}}, Spanner\cite{spanner}, etc.

Desprite of the wide adoption there are a lot of indications that the protocols are complex. Diego Ongaro and John Ousterhout, authors of Raft, wrote in their paper:

\begin{quote}
In an informal survey of attendees at NSDI 2012, we found few people who were comfortable with Paxos, even among seasoned researchers. We struggled with Paxos ourselves; we were not able to understand the complete protocol until after reading several simplified explanations and designing our own alternative protocol, a process that took almost a year
\end{quote}

Google's enginneers wrote about their expirence of building Paxos-based database in the Chubby paper:

\begin{quote}
Despite the existing literature in the field, building such a database proved to be non-trivial \ldots{} While Paxos can be described with a page of pseudo-code, our complete implementation contains several thousand lines of C++ code \ldots{} There are significant gaps between the description of the Paxos algorithm and the needs of a real-world system.
\end{quote}

The complexity of RSM protocols may lead to errors in implementations. Kyle Kingsbury made a comprehensive research\footnote{\href{https://aphyr.com/tags/jepsen}{https://aphyr.com/tags/jepsen}} of distributed consistent databases and found violations of linearizability in almost every database he tested including MongoDB, Etcd, Consul, RethinkDB, VoltDB and CockroachDB.

{\bf Contributions.} I present CASPaxos, a novel protocol for building RSM that avoids complexities of Multi-Paxos and Raft. According to D. Ongaro and J. Ousterhout the complexity of Multi-Paxos comes from the composition rules for instances of the Single-Decree Paxos

\begin{quote}
We hypothesize that Paxos’ opaqueness derives from its choice of the single-decree subset as its foundation \ldots{} The composition rules for Multi-Paxos add significant additional complexity and subtlety.

One reason is that there is no widely agreedupon algorithm for multi-Paxos. Lamport’s descriptions are mostly about single-decree Paxos; he sketched possible approaches to multi-Paxos, but many details are missing. As a result, practical systems bear little resemblance to Paxos. Each implementation begins with Paxos, discovers the difficulties in implementing it, and then develops a significantly different architecture \ldots{} real implementations are so different from Paxos that the proofs have little value
\end{quote}



\section{Overview}
\subsection{Single Decree Paxos}
\subsection{CASPaxos: Intuition}

\begin{appendices}
\section{Proof}
\href{http://rystsov.info/2015/09/16/how-paxos-works.html}{http://rystsov.info/2015/09/16/how-paxos-works.html}
\section{FPaxos}
\end{appendices}

\newpage

\begin{thebibliography}{9}

\bibitem{lamport01}
  Leslie Lamport,
  \emph{"Paxos Made Simple"}.
  2001.

\bibitem{raft}
  Diego Ongaro, John Ousterhout
  \emph{"In Search of an Understandable Consensus Algorithm"}.
  2013.

\bibitem{epaxos}
  Iulian Moraru, David G. Andersen, Michael Kaminsky
  \emph{"There Is More Consensus in Egalitarian Parliaments"}.
  2013.

\bibitem{chubby}
  Tushar Chandra, Robert Griesemer, Joshua Redstone
  \emph{"Paxos Made Live - An Engineering Perspective"}.
  2007.

\bibitem{spanner}
  Corbett, J. C., Dean, J., Epstein, M., Fikes, A., Frost C., Furman, J.J., Ghemawat, S., Gubarev, A., Heiser, C., Hochschild, P., at al.
  \emph{"Spanner: Googles globally distributed database"}.
  2012.


\end{thebibliography}

\end{document}